\documentclass[12pt]{article}
\usepackage{setspace}\doublespacing\usepackage{indentfirst}
\usepackage[left=1in,right=1in,top=1in,bottom=1in]{geometry}
\pagenumbering{gobble}
\begin{document}

The Disputation of 1263 was an example of the relations between Christians and Jews in 13th century Europe.
It shows the main interaction between the two groups is that of the Christians attempting to convert the Jews by both words and eventually force.
Creating a graphic history of the Barcelona Disputation would allow the diverging perspectives of the debate to be explored equally.
This will allow both historians and the general public alike to understand the debate and the events that led to it, and will allow readers to see another side of the perspectives of the people in 13th century Western Europe.

The Barcelona Disputation of 1263 was a theological debate hosted by James I, king of Aragon, between the Dominican Friar Paul and Nachmanides, a prominent rabbi from Girona.
The main subject of the debate was of the nature of the messiah in both religions, with both sides trying to prove their respective cases.
Nachmanides, representing the Jewish consensus, argued that the messiah has yet to come and that Jesus was not the messiah since he was born when the Second Temple still stood.
Paul, representing the Christian clergy, argued that the messiah was none other than Jesus, citing both Bible verses and \emph{aggadot} in his arguments.
This disputation was neither the first nor the last of its kind, and its effects and outcome of the religious debates of the time both happened before and would happen again in the late 13th and early 14th centuries.
The first one of these major debates would take place in in Paris in 1240, led by French King, soon Saint, Louis IX.\@
It ended with the sentencing of the Talmud with its punishment to be round up, burned, and forbidden for Jews to possess.
Like the Barcelona Disputation, there are different perspectives:
``A Latin summary and two Hebrew narratives fo the trial remain; they agree about the basic points of contention but provide very different pictures of what transcribed during the meeting'' (\emph{Debating Truth}, 175).
Like Barcelona, the Latin account is a summary and the Hebrew accounts are narratives that all tell similar stories of the same thing.
There were also disputations that occurred after Barcelona.
Friar Paul would host another Disputation in Paris in the the late 13th century.
For this disputation, one Hebrew source remains (\emph{Debating Truth}, 176).
These disputations largely had a negative effect on the already deteriorating relations between that of Christianity and Judaism.

A prime example of these relations would be from the Fourth Lateran Council, hosted in Rome in 1215.
Some of the Canons of this council would explicitly prevent Jews from holding any form of power in any Catholic land:
``It would be too absurd for a blasphemer of Christ to exercise power over Christians'' (\emph{Debating Truth, The Fourth Lateran Council}, 128).
Jews were forbidden from holding any power in the communities they lived in, and potentially had for generations.
They were also again forbidden from lending any money to Christians, and Christians were again forbidden to borrow money from the Jewish moneylenders.
This would largely go ignored by the nobles, who would still borrow the money when they needed it.
The council also proposes, like the Pact of Umar in Muslim lands, that Jews and Muslims would be forced to wear different clothing and that they were not allowed outside on holy days.
When implemented in Muslim lands, it was already routinely ignored by both Muslims and non-Muslims.
This rule, likewise, would also not be implemented.
One part of the Council that was implemented in Aragon was that of forcing the Jews to attend public sermons hosted by Dominican friars.

This rule was implemented in the land of the Crown of Aragon in 1263, not long after the disputation.
The letter explicitly allowed friars to
``bring Jews and Saracens [Muslims], young and old, men and women, and, if it should be necessary, compel them to meet face-to-face with the same friars, where and when they [the Dominican Friars]  wish'' (\emph{Debating Truth, Document III}, 117).
Empowering the friars to attempt to force the Jews to convert would be largely unsuccessful, but it would be attempted numerous times.
Only three days later, James I sent another letter to tell them to attend the sermons of Friar Paul (\emph{Debating Truth, Document V}, 118).
These orders would be amended the day after that, 30 August 1263, that Friars should ``not compel nor shall you compel nor shall you permit the Jews of your communities, villages, and places of our jurisdiction to be compelled \ldots to go to any location outside their \emph{calla iudayca} [Jewish Ghetto] for the purpose of listening to sermons of any of the preaching friars'' (\emph{Debating Truth, Document VI}, 119).
This sudden turnabout was likely lobbied to James I by the Jewish community of Barcelona and the surrounding area, as friars would go to Synagogues on Shabbat and preach the story of Christ as the messiah to the attendants (\emph{Debating Truth, The Graphic History}, 46--47).

The main primary source to draw from for the new graphic novel is the Latin Account of the Disputation of Barcelona.
Compared to the Hebrew Account, the Latin Account is less of a story and more of a description of an event.
The Latin account is also biased to say that Paul's argument was objectively correct and that Nachmanides either agreed to the points stated or would deny them despite the evidence.
Nachmanides also does not say much in the early part of the account, only agreeing or disagreeing to points stated by the friars.
Friar Paul's points are more complete and more detailed, attempting to display him as more knowledgeable than Nachmanides on both Christian and Jewish sources.
Nachmanides is also portrayed as deceitful and unwilling to face facts:
``he [Nachmanides] was unwilling to admit the truth unless compelled by the authorities, when he could not explain the authorities, he publicly stated that he did not believed in the authorities that were cited against him.'' (\emph{Debating Truth, The Latin Account}, 116).
The end of the Disputation ends the same way, but Nachmanides is much more accepting of the results in the Latin Account, as if he was admitting he was wrong.
It also states that the Rabbi flees Barcelona after his shameful defeat in the Disputation.
This contrasts with the Hebrew Account.

The Hebrew Account is much more detailed on the events of the Disputation.
Nachmanides' perspective of the disputation is much more a dialogue of his perspective then just a description, like the Latin Account was.
Compared to the Latin Account, Nachmanides is an honest man that is in Barcelona because James I summoned him to argue what he believes and was allowed the freedom to say whatever he wants: ``\,`I will do as you command [engage in the Disputation], my lord king, if you give me permission to speak as I please'\,'' (\emph{Debating Truth, The Hebrew Account}, 91).
Instead of being the honest one, Friar Paul instead tries to deceive the authorities by citing sources out of context, which Nachmanides promptly corrects.
The rabbi also is portrayed as more sympathetic in his account, as he knows that the disputation is rigged against him and it is unlikely the king will say that the Jewish perspective is the correct one.
In the end, instead of running away King James allows the rabbi to leave Barcelona and go back to Girona unharmed.

The most difficult part about telling this story again is to minimize the biases in both accounts.
Both sources are naturally biased towards both the authors and the audience, since the writers already believe one side to be the exclusive truth and the other side to be that of liars and frauds.
One possible way to minimize that would be to to use both accounts fully as sources for the new story, but this might still be biased towards one side, likely Nachmanides. 
Another way to minimize the audience would be still using both accounts but minimizing the amount of bias present, but that may not be practical since there are only two sources on the Disputation and they are both biased.
The last way, the way that should be preferred, would be to display both perspectives in similar style.
An example of doing this would be to include that which already exists in the current version of \emph{Debating Truth}, but adding onto it with new information from Friar Paul's perspective.
This will also be a way to show the differences between the two accounts, and to show the differences between the two authors and their perspectives.
The example from before will be a more complete narrative of the story.

The narrative told by Paul's perspective of the Barcelona Disputation would be similar to the story told by Nachmanides', but it will differ in a few ways.
It will include the personal history of the Friar, including that fact that he converted from Judaism to Christianity (\emph{Debating Truth, The Graphic History, 5}).
The added emphasis on Paul's backstory will give the reader perspective on why the Friar knows about Jewish sources: he used to read and believe them before he converted to Christianity.
This will show Paul in the same sympathetic light that Nachmanides was given in \emph{Debating Truth}.
Showing both men as sympathetic and only trying to show the other what they believe to be true will allow the reader to fairly judge the two's perspectives themselves, allowing all arguments to be shown as nothing more than genuine debate.
It will also take some of the more detailed events from the Hebrew Account but will make sure that the bias does not affect the narrative.
Nachmanides will be portrayed similar to the Latin Account, also making sure that he agrees to the points stated in that account.
This new perspective should show readers that like Nachmanides, Friar Paul is a man of knowledge and will show that he is not arguing with the intentions of deception.
Another potential narrative difference could potentially be a sequel story would continue Paul's story to the later Disputations he participated in and his continued preaching to the Jews the nature of Jesus Christ as the messiah.
Paul's insistence that Christ is the messiah told in the Bible and the \emph{aggadot} shows that he is both a man of faith and fully devoted to his cause.
Showing Friar Paul's determination to convert the Jews as he had done in his own life should give readers a new perspective on the Christian side of the story.

One main reason to retell the disputation from the perspective of the Friar Paul instead of the perspective of Nachmanides would be comparing the two perspectives of the accounts.
Both accounts are biased towards the perspective of the authors, so both only show the story in one specific way: the way that makes them look best.
Adding in a second perspective would allow readers to see the differences between the two accounts, and to see the differences between the two authors and their perspectives.
It will allow readers to see that neither side is inherently evil, just that they are both trying to show the other what they believe to be true.
Having the debate as fair as possible will allow non-historians to see that history always has more than one perspective.

Another reason would be to further explain the backstory and of Paul like \emph{Debating Truth} does to explain the backstory and fate of Nachmanides.
The graphic novel part of \emph{Debating Truth} only has one sentence showing that Paul was a Jewish convert to Christianity.
This aspect can be easily missed by many readers, as it is only briefly mentioned on the first page and never brought up again.
The backstory of Nachmanides is much more detailed, and it is also much more complex, comparatively.
Doing the same thing for both members of the Disputation will allow readers exactly why the Disputation occurs.
Sharing what Friar Paul does afterwards will also allow readers to see Paul as more of a person than a character in the story of Nachmanides.
This will show readers that Paul is devoted to the cause that he believes him, making him seem like more of a complete person.

The Barcelona Disputation of 1263 should be used as a prominent example of dealing with two opposing viewpoints and their corresponding biases.
Showing the dueling perspectives of the same event in the same context should inspire readers to see other events in multiple perspectives, either in historical contexts or in their real lives.
Conciliating two opposing viewpoints is an important aspect of the modern age, so showing a historical example will hopefully prevent history from repeating itself.

\end{document}