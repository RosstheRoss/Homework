\documentclass[12pt]{article}
\usepackage{biblatex-chicago}
\bibliography{3053}
\usepackage[T1]{fontenc}
\usepackage{setspace}\doublespacing\usepackage{indentfirst}
\usepackage[left=1in,right=1in,top=1in,bottom=1in]{geometry}

\title{HIST 3053 Paper: \emph{On the Roman Republic and its End}}
\author{Matthew Strapp}
\begin{document}
    \begin{singlespace}
        \maketitle
    \end{singlespace}

    The legendary story of the Roman Republic begins with the expulsion of the tyrannical Lucius Tarquinius Superbus by Lucius Junius Brutus and Lucius Collatinus. Through the near five hundred years of the republic's existence, the senate was the ultimate authority of the Roman government. The senate's authority would wane at the end of its life when strongmen came to power not by popular will but by force. The final deathblow to the structure of the republic would come when Octavian resigns from the traditional highest office of the consulship after serving thirteen times while remaining in power. This bypass of the traditional structure would never be closed through the remaining existence of a Roman state, permanently ending any remaining facade of a republic.

    The legend surrounding the founding of the republic was symbolized by the lifelong rule of the king being replaced with the annual rule of the two consuls. The king's religious duties were also replaced by that of other priests who were subordinate to the chief pontiffs.~\autocite{LivyII} This delegation of authority would be a staple of the Roman Republic and would eventually be a model for modern republics. The consulship would not be firmly established until many years of the republic's existence, but when the nascent republic matured the rule of the consuls would be how the historical years were taken. A autocratic figure would not again be in control of the Roman state for five hundred years.

    The republic would begin its slow end with the Civil War between Marius and Sulla over command in the First Mithridatic War. It would end with Sulla marching on Rome and sentencing his enemies to death. He would then go back to fighting Mithridates only to come back to Rome once again to march on Rome once more. This second march on Rome was different from the previous. Sulla would set up proscriptions for all of his enemies once more. He additionally sanctioned the deaths of all those who support the proscribed, deprive the descendants of the proscribed of their rights as Roman citizens, and would allow the theft of property for all involved.~\autocite[31.4]{Plutarch-Sulla} These proscriptions also extended outside of Rome to the entire Italian peninsula. This unprecedented scale allowed Sulla and his allies to purge many of their rivals. This would set the precedent for both Caesar's crossing of the Rubicon and the Second Triumvirate to proscribe enemies after the death of Julius Caesar.

    Sulla's second march on Rome also allowed him to assume the dictatorship by force in 81 BC.\@ Sulla, using the powers as dictator, proceeded to overhaul the republic into his own image. Many traditionally powerful roles in government were completely curtailed, primarily censors and the plebian tribunes. The tribunes could no longer introduce legislation, and were no longer allowed to hold any office higher up the \emph{cursus honorum}.~\autocite[p. 124]{Flower-Sulla} The censorship was completely abolished. Sulla would write the previously unwritten \emph{mos maiorum} into legal code, including cementing the age requirements of the offices of the republic. He also disallowed anyone from taking the same position for ten years, likely because his rival Marius was consul seven times. After these and many more reforms that were made because Sulla came to power by force, he would resign from the dictatorship after a year. He would then serve as consul before allowing the drastically altered republic before dying soon after. Sulla's reforms attempted to make sure that someone else would not follow Sulla's precedent. This lasted for only twenty years before Caesar would march on Rome like Sulla before him.

    Sulla's restored republic would run into crisis not even ten years after its birth. Lucius Catiline would lead a conspiracy to overthrow the republic that only succeeded due to extralegal means. Catiline lost the election consulship twice in 63 and 62 BC, and would attempt to overthrow the republic. Marcus Tullius Cicero, consul at the time, would famously expose him and his conspiracy in front of the senate. In his speech, Cicero directly tells Catiline that the consuls will put him to death for his treachery.~\autocite{Cicero-Catiline} Catiline would leave the city and attempt to march on the city like Sulla did before. The difference here is that many of Catiline's supporters were executed by Cicero and the Senate, preventing many from joining his cause. Catiline still fought and eventually was slain in battle.~\cite{Sallust-Catiline}. Catiline failed where Sulla succeeded because Cicero executed many of the conspirators without trial through the power of the \emph{senatus consultum ultimum}.

    During this time, philosophers and historians would study the prior republic and attempt to establish both what the republic meant and why it is failing. Cicero would define a republic as the ideal compromise between an Athenian democracy and the monarchy of the Hellenistic states of the Near East. While it seems like the direct rule of the people would be the most effective path of liberty, Athens and other democratic Greek city-states routinely fell to the rule of demagogues who came into power promising the people everything and delivering nothing. Monarchies, on the other hand, can fluctuate wildly because just because one is a good ruler does not mean one's heir will follow that of his father.~\autocite{Cicero-DeRePublica}  Monarchies rarely ever allowed those not born into their positions to take power regardless of merit. The republic allows those of merit and virtue to ascend to power while also preventing the ascension of demagogues because many decisions were made by an elite few, preventing possibly disastrous policies from being implemented.

    Sulla's republic would end when Julius Caesar became dictator for life after fighting another civil war and marching on Rome. Unlike Sulla, Caesar was more forgiving of his enemies. This did not prevent him from having enemies, where he would be famously assassinated in 44 BC.\@ Caesar's death at the hands of the conspirators led by Cassius and Brutus, a claimed descendant of the legendary Brutus that established the republic, would proclaim the restoration of the republic and the death of a tyrant. This move was not popular with the people. Caesar had been popular among the populace, so the announcement of his death caused the people to mobs both at Caesar's funeral and to find and punish the conspirators~\autocite{RomanRevolution}. Caesar would be succeeded by Marc Antony, Caesar's second-in-command at the time of his death, and Octavian, Caesar's adopted son and heir. Any semblance of peace between the Caesareans and the conspirators would be crushed when Octavian would abandon any previous compromise the two camps made and would cause yet another civil war that ended with the deaths of Brutus and Cassius. Like Sulla but unlike Caesar, the new order would once again proscribe their enemies, most famously with Cicero, who was executed on the orders of the Second Triumvirate. Also unlike Caesar and Sulla, instead of being dictator the new triumvirs had extralegal power bestowed upon them.

    The Second Triumvirate eventually fell apart and yet another civil war would break out between Antony and Octavian. This civil war would end with Octavian triumphing over Antony and becoming consul thirteen times.~\autocite{Dio} He would then relinquish the consulship after thirteen but remain in power by having the senate grant numerous privileges to him that allowed him to completely bypass the senate, the consuls, and the traditional structure of Rome. Augustus would finally put the almost dead republic out of its misery by permanently bypassing the republican structure. The senate would progressively wane in power before by the Byzantine era the senate would essentially no longer exist. Never again in the Roman state would the republic in name actually be a republic.

\printbibliography\end{document}
