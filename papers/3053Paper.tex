\documentclass[12pt]{article}
\usepackage{natbib}
\usepackage{setspace}\doublespacing\usepackage{indentfirst}
\usepackage[left=1in,right=1in,top=1in,bottom=1in]{geometry}

\begin{document}
    \begin{abstract}
        The Roman Republic began when Lucius Junius Brutus and Lucius Collatinus overthrew Lucius Tarquinius Superbus and established the rule of the Senate and People of Rome. The Republic fell when the SPQR was no longer \emph{de facto} ruled by consuls but instead who came to power not by election but by force. The final spiral started with Sulla's first march on Rome and finally ended when Octavian permanently resigned from the consulship while still in power. The death rattle of the republic was the era between Sulla and Octavian when the republic cracks under high-pressure situations like the Catiline Conspiracy and the civil war between Pompey and Julius Caesar.
    \end{abstract}
    \section*{Citations and their reasons}
~\cite{DeRePublica} features Cicero's opinions on the Republic while it falls around him. The dialogue about the failings of both monarchy and democracy show the Roman opinions on both while promoting the compromise of republicanism.

~\cite{PlutarchSulla} portrays Sulla marching on Rome and sentencing his enemies to death. His actions set precedent for the men who would finally end the illusions of the republic.

~\cite{RomanRevolution} follows directly after the assassination of Julius Caesar and portrays the will of both his assassins and the people of Rome, who supported Caesar over the restoration of the Republic. The unpopularity of the Liberators shows that even the people wanted the rump Republic to end.

    \bibliographystyle{chicago}
    \bibliography{3053}
\end{document}
