\documentclass[12pt]{article}
\usepackage{setspace}\doublespacing\usepackage{indentfirst}
\usepackage[left=1in,right=1in,top=1in,bottom=1in]{geometry}

\title{HIST 3767 Midterm: \emph{On Iconoclasm}}
\author{Matthew Strapp}
\begin{document}
\begin{singlespace}
    \maketitle
\end{singlespace}
    Iconoclasm, according to the Oxford English Dictionary, means ``The breaking or destroying of images; esp.\ the destruction of images and pictures set up as objects of veneration''\cite{oed:iconoclasm}. Iconoclasm in Orthodox Christianity was the promoted version of the Orthodox religion of the Byzantine Empire twice, from 726 to 787 and from 815 to 843. Both times the iconoclasts were refuted and the iconodules triumphed over their counterparts. The iconoclastic controversy would be one of the major aspects of the split between the East and West that culminated with Leo III crowning a Roman emperor in the west while one reigned in the east.\

    Iconoclasm's roots in the Byzantine Empire trace back to the Roman era of Constantine. During that time, one of Constantine's sisters asked a member of the imperial circle to send her an image of Christ. Eusebius, the member of the imperial circle, refused: \begin{quote}
        Eusebius was evidently prepared to accept images that were merely symbols that pointed to realities beyond themselves. Christ could not be depicted but some of God's mysteries could be revealed by artistic renderings.~\cite[p. 13]{Noble1}
    \end{quote}\noindent
    Icons would remain controversial in the Roman world before the Byzantine iconoclasts of the eighth and ninth centuries. Arguments would be made by men like St.\ Basil and Gregory of Nyssa saw these representations as symbols of what they represent~\cite[pp. 16-17]{Noble1}, while others would see iconodules as sinners practicing idolatry, referencing Biblical verses against such idolatry. Religious issues remained largely arguments between theologians until after the Byzantine Empire went through long periods of internal war and external invasions. The Arab invasions of the Byzantine lands after the rise of Muhammad and his followers combined with the Twenty Years' Anarchy would cause an upending of Byzantine dynasties and ended in 717 with the rise of Leo III.\@ \
    
    Leo III established the new Isaurian dynasty when he acquired imperial power in 717 during a period of near-collapse of the Byzantine empire. There had been six different emperors in the past twenty years, and the Muslims were invading the Byzantine heartland and had laid siege to Constantinople. Soon after taking power Leo had issued proscriptions against all icons. In 727 the emperor ordered the removal of the icon of Christ Antiphonetes taken off of the Chalke Gate in Constantinople. Leo's motivations for doing this seem to stem from Islamic influence. Yazid II, Umayyad caliph at the time, also had initiated an iconoclastic campaign in Islam.~\cite[p. 51]{ByzNotes} This move, while supported by many of the senior members of the clergy and high-ranking military officers, also had significant detractors. \

    Leo III's policies on iconoclasm would be controversial. One source of dissent came from both the Patriarch of Constantinople and the Roman Pope. The Patriarch of Constantinople would eventually be deposed in 730 as neither party was willing to compromise. During this time Leo would issue another edict against icons. Leo's conversations with Pope Gregory II had survived by letters between the two: \begin{quote}
        The issue concerned in part the question of authority. Leo claimed that he was sa priest no less than a king and could legitimately intervene in the running of the Church.~\cite[p. 52]{LeoJour}
    \end{quote}\noindent
    The question of the authority between the Byzantine emperor and Pope would last until the crowning of Charlemagne as emperor of the Romans by Leo III and would become a source of both political and theological conflict in the Catholic world until the Protestant reformation. Leo III did also have political rivals to his iconoclasm.

    Another source of the dissent towards Leo III would come from the theme of Armenia. Armenia also had been iconoclastic before, when it was the frontier between the Byzantines and the Muslims.~\cite[p. 58]{LeoJour} Artavasdus, the \emph{strategos} of Armenia during Leo's rise to power, had been given a theme of his own and the hand of one of the emperor's daughters in marriage. The theme would still be opposed to Leo after his usurpation. His opposition to Leo III would extend beyond political matters and would support the iconodules, sheltering them from the iconoclasts. Artavasdus opposed the succession of Leo's son, Constantine V. \

    Constantine V's reign would feature more iconoclasm than that of his father. His iconoclastic tendencies would not show during the beginning of his reign due to fighting back an Umayyad invasion, a campaign in the Balkans, and financial reform. It was soon after fending off the Byzantine Empire from collapse that Constantine issued an edict ordering the destruction of icons. Constantine also called a council similar to his namesake in Nicea in 754. This second Nicean Council was attended by iconoclastic bishops from all over the empire. These bishops were appointed by Leo III and Constantine V precisely because they supported the iconoclasts. They were not puppets, either: \begin{quote}
        The assembled bishops should not be seen as imperial lackeys bu instead as men who had made their own the church's longstanding uneasiness about religious art. There is no evidence for violence or threats.~\cite[p. 63]{Noble2} 
    \end{quote}\noindent
    The council outlawed depictions of the icons and would name and anathematize three major iconodules: Germanus, the patriarch Leo deposed in 730, John of Damascus, a man who wrote orations supporting the veneration of icons, and George of Cyprus, a key figure in the admonition of the iconoclasts before Leo III empowered them.~\cite[p. 64]{Noble2}  The council also explicitly stated the official policy when it came to icon worship and portrayal: \begin{quote}
         `every icon, made of any matter \dots is objectionable, alien and repugnant to the Church of the Christians' ~\cite[p. 64]{Noble2}
    \end{quote}\noindent
    All who made or venerated an icon were to be cast from the church. There were also checks in place, though. Alteration of sacred vessels was heavily restricted, and laymen were not allowed to vandalize any church property. All edits done were approved by higher ups, to prevent overzealous people from destroying churches like what happened during the Protestant Reformation or the French Revolution. This council was protested by the pope, who urged Constantine to restore the icons. These calls were ignored. After the council, the emperor soon started persecuting all those found to be venerating icons by excommunicating them from the Church. Much of this persecution was directed at the monastic orders. Many monks still venerated icons even after the Second Nicean Council explicitly forbade them. Constantine would force them monks to be humiliated in public for not following directives. \

    John of Damascus, who had died before being condemned in Nicea, would be instrumental in refuting the first wave of Iconoclasts.\footnote{Much of this section is paraphrased from~\cite{JohnDam}.} John's defense of the icons stemmed from different kinds of worship: \begin{quote}
        An important distinction that John makes is that between absolute worship \dots and relative worship.~\cite[p.457]{JohnDam}
    \end{quote}\noindent
    This difference, with the latter usually used for prayer to holy sites and the former only to God, refutes an iconoclast point about there only being one kind of prayer. John also supposed that since God created man in His own image, paying honor to others is similar to honoring God. John also states that when people worship an image of Christ they are not worshipping that image but what the image represents. His splitting of the image from what it represents will be one of the main reasons of the reinstating of icons at the Seventh Ecumenical Council. For his efforts he is now venerated as a Saint in both the Catholic and Orthodox churches. \

    After Constantine V's death in Bulgaria, iconoclasm continued but without the fervor of the emperor behind them. Art around the empire depicted icons was replaced with more secular art instead. This was applied more in the Greek parts of the empire than the Italian parts, where early Byzantine iconography can still be found in cities like Ravenna. Leo IV, Constantine's son, was an iconoclast but not as hardcore as that of his grandfather or father. His reign of the Byzantium mainly consisted of fighting the Abbasid Caliphate in Syria. These campaigns proved modestly successful, but would end abruptly in 780 with his death. After his death, his young son Constantine, now Constantine VI, was emperor. His mother, the empress Irene, acted as his regent. \

    Irene, unlike her husband and his ancestors, was an iconodule. She would soon reinstate icon worship and call for another Ecumenical Council. This was controversial among the iconoclasts that Leo III and Constantine V had filled in the imperial government. She also reached out to the Pope in the west along with the new king of the Franks, Charles. Soon after, the Patriarch of Constantinople had fallen ill and soon abdicated power. Irene took this opportunity to place an iconophile in this important clerical position. Tarsius, the man Irene appointed to be Patriarch, agreed with Irene that another council should be called. \
    
    In 787, the Seventh Ecumenical Council was called in Nicea, the same location as the first Ecumenical Council that Constantine ordered.\footnote{To ease possible confusion, any references to the  `second Nicean Council' or anything similar will refer to the Seventh Ecumenical Council of 787, not the iconoclastic one.}
    The council was established to once again discuss icons and their worship. In comparison to the previous council, clergy from all Chalcedonian Christians would be there instead of only iconoclasts. One of the major members of the council, Epiphanius the Deacon, would set the major differences between the icon from the image it represents: \begin{quote}
         `When one looks at the icon of a king, he sees the king in it. Thus, he who bows to the icon bows to the king in it, for it is his form and his characteristics that are on the icon. And as he who reviles the icon of a king is justifiably subject to punishment for having actually dishonoured the king' ~\cite[p. 129]{IconIdolatry}
    \end{quote}\noindent
    Irene's position as Empress-Regent was always fragile but became even more so refuting the iconoclastic rule that has existed throughout the Byzantine clergy since soon after the rise of the Isaurian dynasty in 717. She would repeatedly face opposition from both secular figures and religious ones throughout the empire that would eventually culminate with the blinding and therefore the deposition of her son Constantine VI, the Papal crowning of Charles in the west as Roman emperor, and Irene's deposition in 802. The first Iconoclastic period started and ended with the Isaurian dynasty. \

    The first period of Iconoclasm, established by Leo III after a period of turmoil throughout the Byzantine Empire, ended with the Second Nicean Council in 787. The era that follows in Byzantine history is largely uneventful compared to the previous eras, but the comparison between little happening and near-collapse makes uneventful times stable. This era also had some of Byzantium's neighbors successfully convert to Christianity, culminating with Bulgaria in 864.\cite[p.12]{BriefByzantine} Bulgaria was important because it also recognized the primacy of the Patriarch of Constantinople. The Serbians and eventually the Rus would also convert in the next century, which would become instrumental in the long-standing life of the Orthodox Church. Before all of this, another Byzantine emperor would ignore the Seventh Ecumenical Council and reinstate iconoclasm. This would be spearheaded by the efforts of Leo V, emperor from 813 to 820. \

    Leo V's adoption of iconoclasm was reliant on Constantine V's military victories in the past being associated with his fervent iconoclasm. Since Leo was in the middle of a war with the Bulgarians, he instated a ban on icons to emulate past victories.~\cite[p. 130]{IconIdolatry} Leo V called another council in 815 that once again banned images. Like his predecessors the Second Commandment was the driving point against icons as portrayals were considered idolatry. This was again controversial like before but this was even more so than the previous iconoclastic period: \begin{quote}
        [B]y this time iconophilia had been official policy for decades. The patriarch was a known iconophile as were the influential Studite monks arranged behind their leader, [the future Saint] Theodore.~\cite[p. 247]{Noble6}
    \end{quote}\noindent
    Theodore would soon write his famous \emph{Refutations of the Iconoclasts} but that was not until after he was exiled by the once again iconoclastic clergy. The Patriarch of Constantinople once again deposed by the emperor and was replaced by an iconoclast. The new patriarch was also a nephew of Constantine V. This new council would routinely praise the Isaurians and would blame the return of iconodule rule on Irene. This reinstating of iconoclasm was followed by Byzantine victories against both the Muslims and the Bulgarians, convincing some who remained on the fence before. After the council and Theodore's exile, he will start his work on the icons that would eventually become the standard Orthodox works on the icons. \

    St.\ Theodore, along with Nicephorus, the ousted Patriarch of Constantinople, would also publish works refuting iconoclasm, like John of Damascus before them: \begin{quote}
        Nevertheless, their [Theodore and Nicephorus] writings were not merely refutations of what had been said by iconophobes. They also formulated new doctrines and developed new approaches to the whole subject. Together the represent the high point of Byzantine theological reflection on the question of images.~\cite[p. 252]{Noble6}
    \end{quote}\noindent
    Theodore's \emph{Refutations of the Iconoclasts} was one of the prominent works that Theodore made during this exile. In this work, written as a dialogue between an iconodule, labelled as Orthodox, and an iconoclast, labelled as Heretic. He has similar points that John of Damascus had, including separating the symbol from what it represents: \begin{quote}
        This principle also applies to Christ and His icon. For Christ is called `very God' and also `man', because of the signification of name, but not because it has the nature of divinity and humanity.~\cite[p. 52]{OntheIcons}
    \end{quote}\noindent
    These refutations would be labelled as heretical by the iconoclasts but would soon become seminal works of Orthodox literature. \

    The second triumph of the iconodules over the iconoclasts did not occur until after the death of Theophilos in 842, after he and Michael II, Leo V's successor, persecuted iconodules. Similar to Irene, Michael III was emperor as a child and as such was under the regency of his mother, Theodora. Theodora would work with the exiled monks, following Theodore's (who had died in 826) teachings to restore the veneration of icons. Like Irene, Theodora restored icon worship while acting as the regent of a child. Veneration was restored in March of 843.~\cite[p. 112]{Herrin} Unlike Irene, Theodora did not blind her son in a power grab that ended up alienating the west and crowning of two Roman emperors. Theodora is now a saint in the Orthodox Church for her final work against the iconoclasts. \

    The iconoclasts would never take power again, and the church would move to Russia after the Byzantine Empire falls in 1453. The iconoclastic controversies would be one of the causes of the East-West schism that fully formulated in 1054. The iconoclasts would successfully destroy Byzantine art in the heart of the empire such that little of the art from the times before Leo III exist. It also helped the Orthodox church by establishing what is simple icon worship versus the sin of idolatry. The triumph of Orthodoxy over iconoclasts would cement the Byzantine Empire to live for another six hundred years where before Leo the Isaurian the empire was on the brink of collapse.

    \bibliography{3767}
    \bibliographystyle{unsrt}
\end{document}