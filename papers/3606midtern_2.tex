\documentclass[12pt]{article}
\usepackage{setspace}\doublespacing\usepackage{indentfirst}
\usepackage[left=1in,right=1in,top=1in,bottom=1in]{geometry}
\pagenumbering{gobble}
\begin{document}

The Crusades facilitated an interaction between the Muslim and Christian worlds brought on by conquest.
These new kingdoms faced challenges surrounded by Muslim ruled neighbors, as well as challenges from the Muslim population inside the crusader states.
Many Christians would eventually start to blend their native Frankish culture along with the native Muslim cultures of their subjects to create a fusion of two separate worlds.

One of the first things that happened after the creation of the crusader states was the beginning blend of two distinct cultures.
The Franks saw the Muslims as outsiders and lesser than them, and many Muslims saw the Franks as oafish overlords that only cared for themselves.
Some Franks, especially new arrivals, were especially confused by the Muslims praying:
``One day I entered this mosque, repeated the first formula, `Allah is great,' and stood up in the act of praying, upon which one of the Franks rushed on me, got hold of me and turned my face eastward \dots A group of Templars hastened to him, seized him and repelled him from me''
(ibn Munqidh, p. 290).

The Knights Templar and other Franks that had been in the Holy Land for a long time were more tolerant of the Muslims than the newer Franks, who would see their religious practices as wrong.
The new Franks had likely never seen a Muslim before, and they were not familiar with the Islamic faith or its practices.

The converse was true for the Muslims as well.
Few if any Muslims knew of the practices of the Western European crusaders.
The Muslims were more tolerant of the Crusaders, but still saw many of their practices as barbaric.
This is best displayed with their medical practices:
``We had in our country a highly esteemed knight who was taken ill and was on the point of death \dots When the priest saw the patient, he said `Bring me some wax.' We fetched him a little wax \dots and he stuck one in each nostril. The knight died on the spot. We [the Muslims] said to him [the priest], `He is dead.' `Yes,' he replied, `he was suffering great pain, so I closed his nose that he might die and get relief''' (ibn Munqidh, p. 293).
The Muslims' reaction to this and the rest of the medical practices seen in the Levant was nothing short of both amazement and horror.
The relations between the Muslims and the Crusaders was also expressed with taxes.

The new crusader states had to establish themselves in a majority-Muslim world.
Some cities would largely treat both religions as equals, sharing crops and animals.
Others would treat the Muslims as second-class citizens in crusader lands.
This was largely done with a similar but opposite tax on goods on the opposing religion.
``The Christians impose a tax on the Muslims in their land which gives them full security; and likewise the Christian merchants pay a tax upon their goods in Muslim lands. Agreement exists between them, and there is equal treatment in all cases'' (Ibn Jubayr, p. 215).
The taxes angered many Muslim merchants who were not used to paying these new taxes.
This especially effected the Maghrebi, who were taxed extra.
``The greater part of those taxed where Maghrebis, those from all other Muslim lands being unmolested. This is because some earlier Maghrebis had annoyed the Franks'' (Ibn Jubayr, p. 216).
This punitive tax was levied on them because the Maghrebis travelled to attack the Crusaders along with the native Muslims of the Levant.
Many Christian rulers also had Muslims in leading positions in their respective lands.
This was done to calm their subjects, as a Muslim is more willing to hear the problems of a Muslim than a Christian.

The crusader states, established by the bloody conquest of the holy land from its Muslim inhabitants and overlords, brought two separate worlds together in a few small kingdoms and counties.
The Franks and Muslims would view each other as outsiders, but eventually their cultures would begin to blend together and become something distinct.

\pagebreak

The Black Plague was devastating as it spread throughout Europe.
During these years, many would search for someone to blame for their misery.
The blame would end up being shifted onto the Jews in the area.
This was largely stemmed from previous anti-Jewish accusations and the views of relative success and unharmed populations compared to the Christians by both the lower and upper classes.

One of the major events of anti-Jewish violence before the Plague was during the First Crusade.
In 1096, crusaders in the Holy Roman Empire would launch a crusade against the Jews in the cities.
``In the spring of 1096 bands of crusaders – poor people as well as experienced knights – attacked and injured the Jewish communities of Speyer, Mainz, Worms, Cologne, Metz, Trier, Regensburg and Prague. Jews were massacred, their property was despoiled and destroyed'' (Bronstein, p. 1268)
The contemporary historians note that both the poor and rich would engage in these massacres, and would emphasize forced conversions.
This was not shared with later historians.
Later historians see the massacres as less of forced conversions and more of rampant looting and murder among the poor.
Others argued that both the richer and the poorer Christians engaged in this and other violence with another motivation: that of economics.

Throughout the middle ages moneylending to Christians was seen as a sin by the Church, so Christians were forbidden from engaging in moneylending.
This was not extended to the Jews.
Many Jews would end up as lenders to those both rich and poor.
The nobility in turn would protect them in return for giving them this money:
``Christian dependence upon Jewish moneylenders constituted a major irritant in Jewish-Christian relations in the High and later Middle Ages'' (Cohen, p. 84).
Much anti-Jewish violence was instigated by the European nobles and the poor who were indebted and wanted to be forgiven by force.
The Catholic Church was also opposed to moneylending, and would attempt multiple times to force all the monarchs of Europe to ban moneylending in every kingdom.
This would largely come to a boil as the Black Plague began to spread throughout Europe from the east.

The Black Death devastated Europe in the 14th century. By the end of the plague, more than a third of the population was dead:
``In the year 1349 there occurred the greatest epidemic that ever happened. 
Death went from one end of the earth to the other, on both sides of the sea, 
and it was even greater among the Saracens than among the Christians'' (von 
Königshofen, p. 155).
The disease affected all, Christians and Jews alike.
Some were not convinced that this plague was natural.
Eventually it was spread through Europe that the group to blame on the disease was the Jews in the area.
They were accused of all sorts of heinous crimes:
``when in anticipation of, or shortly after, out-breaks of plague Jews were accused of poisoning food, wells and streams, tortured into confessions, rounded up in city squares or their synagogues, and exterminated en masse'' (Cohn, 4).
These accusations were rarely tried in a court and those that were tried were usually found guilty through duress confessions or other forms of coercion, including loaded trials.
Cohn argues that instead of the peasants instigating the violence it instead was the nobles and aristocrats who were largely the perpetrators.
Few debters were poor, and many were rich.
The rich were the ones who were most likely to be the ones to blame, since the poor were rarely in debt, unlike the nobles.

Einbinder goes into more specifics on specific groups that were affected by the violence.
His emphasis on the graves in Tàrrega emphasizes the violence done to the Jews: 
``There is more evidence that the dead were buried hastily and in unusual circumstances. Jewish burial practice calls for the body to be garbed only in a shroud and unaccompanied by personal possessions. Yet these individuals were clothed at the time of burial, and a fascinating trove of objects was recovered with their remains, including coins and buttons, buckles and jewelry, a thimble, and the cover of a decorative box'' (Einbinder, p. 131).
These hasty, delayed burials were likely also done by Jews, since they were buried following some of the Jewish burial customs.
Another thing Einbinder emphasizes is the trauma on many of the remains.
A majority of those in graves had traumatized remains.
His last point is that he agrees with other historians that the violence was more caused by the aristocrats then the peasants.

Throughout the 14th century, the Black Plague was used as a \emph{casus belli} against the local Jewish populations. This was largely instigated not by the peasants but by the aristocrats, nobles and the wealthy.
These massacres were largely caused by similar reasons to previous massacres, which were largely economic instead of religious.
\end{document}