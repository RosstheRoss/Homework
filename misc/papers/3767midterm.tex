\documentclass[12pt]{article}
\usepackage{setspace} \doublespacing
\usepackage{indentfirst}
\usepackage[left=1in,right=1in,top=1in,bottom=1in]{geometry}

\title{HIST 3767 Midterm: \emph{On Iconoclasm}}
\author{Matthew Strapp}
\begin{document}
\begin{singlespace}
    \maketitle
\end{singlespace}
    Iconoclasm, according to the Oxford English Dictionary, means ``The breaking or destroying of images; esp. the destruction of images and pictures set up as objects of veneration''\cite{oed:iconoclasm}. Iconoclasm in Orthodox Christianity was the promoted version of the Orthodox religion of the Byzantine Empire twice, from 726 to 787 and from 815 to 843. Both times the iconoclasts were refuted and the iconodules triumphed over their counterparts. (FINISH) \

    Iconoclasm's roots in the Byzantine Empire trace back to the Roman era of Constantine. During that time, one of Constantine's sisters asked a member of the imperial circle to send her an image of Christ. Eusebius, the member of the imperial circle, refused:
    \begin{quote}
        ``Eusebius was evidently prepared to accept images that were merely symbols that pointed to realities beyond themselves. Christ could not be depicted but some of God's mysteries could be revealed by artistic renderings.''
        \cite[p. 13]{CarlIcons}
    \end{quote}
    \noindent
    Icons would remain controversial in the Roman world before the Byzantine iconoclasts of the eighth and ninth centuries. Arguments would be made by men like St. Basil and Gregory of Nyssa saw these representations as symbols of what they represent \cite[pp. 16-17]{CarlIcons}, while others would see iconodules as sinners practicing idolatry, referencing Biblical verses against such idolatry. Religious issues remained largely arguments between theologians until after the Byzantine Empire went through long periods of internal war and external invasions. The Arab invasions of the Byzantine lands after the rise of Muhammad and his followers combined with the Twenty Years' Anarchy would cause an upending of Byzantine dynasties and ended in 717 with the rise of Leo III. \
    
    Leo III came to imperial power in 717 during a period of near-collapse of the Byzantine empire. There had been six different emperors in the past 
    twenty years, and the Muslims were invading the Byzantine heartland and had laid siege to Constantinople. Soon after taking power Leo had issued proscriptions against all icons. In 727 the emperor forced the removal of the icon of Christ Antiphonetes taken off of the Chalke Gate in Constantinople. \cite[p. 51]{LeoJour} The Umayyad caliphate had also banned icons around 721-723, before Leo III banned icons.

    \bibliography{3767}
    \bibliographystyle{unsrt}
\end{document}