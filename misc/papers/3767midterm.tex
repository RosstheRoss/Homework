\documentclass{article}
\usepackage{setspace} \doublespacing
\usepackage{indentfirst}
\usepackage[left=1in,right=1in,top=1in,bottom=1in]{geometry}

\title{HIST 3767 Midterm: On Iconoclasm}
\author{Matthew Strapp}
\begin{document}
\begin{singlespace}
    \maketitle
\end{singlespace}
    Iconoclasm, according to the Oxford English Dictionary, means ``The breaking or destroying of images; esp. the destruction of images and pictures set up as objects of veneration''\cite{oed:iconoclasm}. Iconoclasm in Orthodox Christianity was the promoted version of the Orthodox religion of the Byzantine Empire twice, from 726 to 787 and from 815 to 843. Both times the iconoclasts were refuted and the iconodules triumphed over their counterparts. (FINISH)


    Iconoclasm's roots trace back to the era of Constantine. During that time, one of Constantine's sisters asked a member of the imperial circle to send her an image of Christ. The member refused:
    \begin{quote}
        ``Eusebius was evidently prepared to accept images that were merely symbols that pointed to realities beyond themselves. Christ could not be depicted but some of God's mysteries could be revealed by artistic renderings.''
        \cite{CarlIcons}
    
    \end{quote} 
    \bibliography{3767}
    \bibliographystyle{unsrt}
\end{document}