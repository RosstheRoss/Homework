\documentclass{article}
\usepackage{setspace} \doublespacing
\usepackage{indentfirst}
\usepackage[left=1in,right=1in,top=1in,bottom=1in]{geometry}

\title{HIST 3767 Midterm: On Byzantine Iconoclasm}
\author{Matthew Strapp}
% \bibliography{3767}
% \bibliographystyle{chicago-authordate}
\begin{document}
\begin{singlespace}
    \maketitle
\end{singlespace}
    Iconoclasm, according to the Oxford English Dictionary, means ``The breaking or destroying of images; esp. the destruction of images and pictures set up as objects of veneration''\cite{oed:iconoclasm}.

    Iconoclasm had its roots in Islamic teachings but became prominent in Byzantine politics after Leo III had removed the icon of Chirst Antiphonetes at the Chalke Gate in Constantinople around 726. \cite{LeoJour} 

    \bibliography{3767}
    \bibliographystyle{abbrv}
\end{document}