\documentclass[12pt]{article}
\usepackage{fullpage}\usepackage{indentfirst}

\begin{document}
    \section{Question 1}
    \subsection{Overlapping Intervals}
    \subsection{Non-Overlapping Intervals}
    \section{Question 2}
    \subsection*{A}
    Shared memory has all of the processors access one large pool of memory while distributed memory has each processor have a section of the memory.
    \subsection*{B}
    Distributed memory programs communicate via pipes and message queues while shared memory programs share memory with locks in place to prevent unintended behavior and race conditions.
    \subsection*{C}
    Shared memory is in most devices nowadays with even phones having multiple cores and a single bank of shared memory. Large rendering farms and HPC will have distributed memory to split rendering or to maximize performance.
    \subsection*{D}
    Distributed is easier to scale because the cluster can have nodes added or removed easily without affecting the other nodes.
    \section{Question 3}
    \subsection*{A}
    Concurrency: 8\\
    Critical path length: 4\\
    Maximum achievable speedup: \(15/4\) \\
    Minimum number of processes needed: 8\\
    Maximum achievable speedup if the number of processes is limited to: \\
        2- \(15/8\)\\
        4- \(15/5\)\\ 
        8- \(15/4\)\\
    \subsection*{B}
    Concurrency: 8\\
    Critical path length: \\
    Maximum achievable speedup: \\
    Minimum number of processes needed: \\
    Maximum achievable speedup if the number of processes is limited to: \\
        2- \(15/8\)\\
        4- \(15/5\)\\ 
        8- \(15/4\)\\
    \subsection*{C}
    Concurrency: 8\\
    Critical path length: 7\\
    Maximum achievable speedup: \(14/7\)\\
    Minimum number of processes needed: \\
    Maximum achievable speedup if the number of processes is limited to: \\
        2- \(14/10\)\\
        4- \(14/8\)\\ 
        8- \(14/7\)\\
    \subsection*{D}
    Concurrency: 2\\
    Critical path length: 8\\
    Maximum achievable speedup: \(15/8\)\\
    Minimum number of processes needed: 2\\
    Maximum achievable speedup if the number of processes is limited to: \\
        2- \(15/8\)\\
        4- \(15/8\)\\ 
        8- \(15/8\)\\
    \section{Question 4}
    \subsection*{A}
\end{document}