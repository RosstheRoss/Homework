\documentclass[11pt]{article}
\usepackage{fullpage}
\usepackage{times}

\begin{document}
\begin{center}
CSci 5271 Fall 2021 Exercise Set 3 answers template
\end{center}

Please use this document as a template for submitting your answers to
exercise set 3. (This template is available from the course web site
in either LaTeX or Google Doc formats). Type your answers on each page
after the question prompt (you can use additional pages, though that
we expect that would rarely be required). If you can write all your
answers electronically, please do so and export to a PDF to submit.
If you would prefer to hand-draw figures, you can also submit a scan.

Please ensure that the names and UMN email addresses of all of your
group members are recorded on Gradescope, and also confirm them below:

\vspace{10pt}

\begin{tabular}{|p{2.6in}|p{2.6in}|}\hline
Name & UMN email address\\\hline
Matt Strapp & strap012@umn.edu\\\hline 

\end{tabular}

\vspace{10pt}

Question 1 (Caesar's block cipher, 30 pts):

Part 1(a), CCEA2 $>$ CCEA1?

No. An 8-bit block size can be easily broken, regardless of key size.

\vspace{0.5in}

Part 1(b), any 8-bit-block block cipher


\newpage

Part 1(c), ECB mode

\vspace{1.5in}

Part 1(d), CTR mode

\vspace{1.5in}

Part 1(e), CBC mode

\vspace{1.5in}

Part 1(f), OFB mode

\newpage

Part 1(g), weakness of CCEA3

\newpage

Question 2 ((Mis-)using message authentication codes, 26 pts):

Part 2(a), CBC-MAC

\vspace{3in}

Part 2(b), hashing and AES-CTR

\newpage

Question 3 (Protocol (an)droids, 24 pts):

Part 3(a), simpler attack

\vspace{3in}

Part 3(b), second attack

\newpage

Question 4 (Hashing and signing, 20 pts):

\end{document}
